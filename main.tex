\documentclass[12pt,a4paper]{article}
\usepackage{enumitem}
% Define margins
\setlength{\topmargin}{-1.0cm}
\setlength{\oddsidemargin}{0.1cm}
\setlength{\textwidth}{16.5cm}
\setlength{\textheight}{23.0cm}

% Use Times New Roman font
\usepackage{times}
\usepackage{xurl}
\usepackage[hidelinks]{hyperref} 
\urlstyle{rm}

\renewcommand{\rmdefault}{ptm}

\usepackage{graphicx} % LaTeX package to import graphics
\graphicspath{{images/}} % Configuring the graphicx package

% Define header and footer
\usepackage{fancyhdr}
\pagestyle{fancy}
\fancyhf{}
\rhead{\textbf{\textit{Week 5 Submission}}}
\cfoot{\textbf{\textit{\thepage}}}
\renewcommand{\headrulewidth}{0.7pt}
\setlength{\headheight}{14pt}

% Adjust section and subsection title formats
\usepackage{titlesec}
\titleformat{\section}
  {\normalfont\fontsize{14}{15}\bfseries}{\thesection}{1em}{}
\titleformat{\subsection}
  {\normalfont\fontsize{12}{15}\bfseries}{\thesubsection}{1em}{}

% Define a style with no footer for the table of contents
\fancypagestyle{nofooter}{%
  \fancyfoot{}%
}

% To manage references
\usepackage{natbib}
\usepackage[labelfont=bf]{caption}

\begin{document}

% TITLE PAGE

\begin{titlepage}

\newcommand{\HRule}{\rule{\linewidth}{0.5mm}}
\center

\vspace*{1\baselineskip}
\includegraphics[width=0.15\textwidth]{images/UTS.png}\\
\textsc{\LARGE University of Technology Sydney}\\[2.0cm]
\textsc{\Large (32557) Enabling Enterprise Information Systems}\\[0.2cm]

\HRule\\[0.6cm]
{\huge\bfseries E-Business \& E-Commerce}\\[0.4cm]
\HRule\\[10cm]

\emph{by Team Super} \\
{ Seoyoon Kim (25388442) [Group leader] \\}
{ Jin Lee (25388733)  \\}
{ Ariel Manueke (25207919) \\}
{ Nonthawat Praisompong (25233750) \\}

\vfill
{\large\today}

\vfill

\end{titlepage}

% TABLE OF CONTENTS

\tableofcontents
\thispagestyle{nofooter}
\cleardoublepage

\pagebreak

% DOCUMENT CONTENT STARTS HERE
% You can start writing your document content here.


% Student %%%%%%%%%%%%%%%%%%%%%%%%%%%%%%%%%%%%%%%%%%%%%%%%%%%%%%%%%%%%%%%%%%%%%%%%%%%%%%%%%%%%%%%%%

% Seoyoon %%%%%%%%%%%%%%%%%%%%%%%%%%%%%%%%%%%%%%%%%%%%%%%%%%%%%%%%%%%%%%%%%%%%%%%%%%%%%%

\setcounter{page}{1}

\section{Question 1}
\subsection{Digital Inclusion: Strategies for E-Commerce Equity}

The ascendance of e-commerce has reconfigured the commercial landscape, presenting both opportunities and challenges. The individuals who lack digital literacy and resources are confronted with the potential of exacerbated socio-economic exclusion. This phenomenon, often termed the ``digital divide,'' affects not only access to commercial platforms but also has broader implications for employment and essential services participation.\\ 

\noindent In light of this digital divide, it is incumbent upon communities to devise inclusive strategies that ensure equitable participation in the burgeoning digital economy. \citep{Ref1.3} delineate the influence of demographic variables on digital technology access and utilization. Their research underscores the need for interventions to be demographically sensitive, suggesting that community programs should be designed with an understanding of the varied needs across different population segments.\\

\noindent The COVID-19 pandemic has intensified the reliance on digital connectivity, rendering it a lifeline for many. \citep{Ref1.2} articulate how the pandemic has spotlighted the digital divide, necessitating immediate and targeted actions to bridge this gap. Community-driven initiatives, such as the provision of public Wi-Fi networks, could alleviate the burden of individual internet access costs, thus fostering greater digital inclusivity.\\ 

\noindent Furthermore, the notion of digital capital, as expounded by \citep{Ref1.1}, extends beyond mere access to encompass the competencies and actual uses of digital technologies. This perspective advocates for a multifaceted approach to digital inclusion, emphasizing the role of skills and knowledge in leveraging digital technologies effectively. In this regard, IT students have a significant role to play. Through mentorship and the facilitation of skill development workshops, they can impart critical e-commerce navigation competencies, thereby empowering individuals to fully engage with the digital economy.\\

\noindent In conclusion, while the digital divide presents substantial barriers to equal participation in e-commerce, community-focused educational initiatives, bolstered by recent insights into the multifactorial nature of digital exclusion, can cultivate an environment that promotes digital equity. 

\pagebreak%%%%%%%%%%%%%%%%%%%%%%%%%%%%%%%%%%%%%%%%%%%%%%%%%%%%%%%%%%%%%%%%%%%%%%%%%%%%%%

\setcounter{page}{2}

\section{Question 2}
\subsection{The Effect of Two M-Commerce Applications in People's Life and Businesses}
\label{sec:Question 2}
\textbf{Financial Services: Apple Pay}\\
\noindent Apple Pay is an online payment method introduced by Apple in 2014 to make payment easier by simply holding the iPhone near the payment reader to complete any purchase cashless. Moreover, this payment method and handled an impressive 9 out of 10 mobile transactions after only 3 years since it launched \citep{Ref2.1}. The first effect is the bringing the cashless culture, where people starting to leave behind the habit of bringing thick and even physical wallets. This helps many people to have lighter carry-on weight and reduces the risk of losing or having their wallet stolen along with all the important cards and personal information, as all the customer transaction data is safely stored and never leaves the device. Businesses also get a huge impact from this culture, as they start to replace people with robots, for example, fast food restaurants in Australia, only use a touch screen to complete any payments because there’s no need for money change return as the conventional one. Storing all payment data on an online system also helps track spending for customers and makes easier transaction reports for businesses.\\

\noindent\textbf{Accessing Information: Apple Music}

\noindent Apple Music has shown significant growth in the last couple of years due to its strong user base and Apple’s free 3-month trial move to attract customers for their premium streaming music platform \citep{Ref2.2}. Apple Music changed how people listen to music where its users can stream and download their favourite songs using their Apple devices at anytime and anywhere with unlimited choices of music playlists and artists with just a single subscription. This also impacts business in the music industry, because it allows more income for music artists and labels due to the easiness of publishing new releases, and the royalties from every stream that can be monetized without fully relying on traditional record sales such as CD and vinyl. More in-depth look we can see that with real-time and up-to-date releases, all businesses such as retail stores, event venues, restaurants, etc., can play the latest licensed music for commercial purposes to enhance their customer experiences.

\pagebreak
%%%%%%%%%%%%%%%%%%%%%%%%%%%%%%%%%%%%%%%%%%%%%%%%%%%%%%%%%%%%%%%%%%%%%%%%%%%%%%

\setcounter{page}{3}

\section{Question 3}
\subsection{Exploring Factors Influencing Customer Choices in the New Car Market}
\label{sec:Question 3}

\noindent According to \cite{Ref3.1}, It is obvious that customers have been provided options for new cars that they want to buy. The authors also mentioned 2 approaches of customers for selecting or buying a new car. Firstly, the characteristics of the car consist of the engine, model, vintage and how the car was made. Secondly, aggregate representation is the engine size and fuel type.\\

\noindent All customers have different purposes for purchasing their car. However, to ensure they will select the best choice, they must know exactly the car information to find the suitable choice that matches their specific purpose. \\ 

\noindent In this assignment, we selected the Lexus LC 500 as the car we want to purchase, and we will show the reader significant car information that enhances customers' decision-making. Including: \\

\noindent\textbf{Details, specification, features: }\\
\noindent To buy a new car, First, customers always consider the car's specifications. What capabilities can cars have, such as engines, seats, driving systems, fuel consumption, technology, etc. Various details have different perspectives of each person. For example, customers who buy off-road and sports cars think of different purposes. The second-hand buyer needs to check the car history to ensure the condition of maintenance and services including accidents according to \cite{Ref3.3}.\\ 

\noindent\textbf{Price and model: }\\
\noindent According to website \cite{Ref3.2}, some car models have sub-models. LC 500 has 3 different models, including LC luxury, convertible, and hybrid. However, these models have different prices depending on customer customisation.\\

\noindent\textbf{Car safety and warranty:}  

\noindent All customers are concerned about car safety standards and features. When customers compare the cars in the same segment, Safety is one of the important factors that persuade them to buy. Also, warranty is a crucial part of consideration. On the other hand, how much cost can they save from the start until they reach the kilometres that the warranty has been covered? Customers can check the details from the website \cite{Ref3.2}.\\

\noindent\textbf{Privilege: }\\
\noindent Lexus customers receive the benefit from the car company in different ways following this list:
\begin{itemize}
    \item Capped Price Service: Estimate the price of services fee in each service schedule.
    \item Lexus Drive Care: Lexus provides 24/7 assistance for customers with car breakdowns.
    \item Service Car Loan: Lexus provides a reserved car for customers who take a service.
    \item Discount: Travel and fine dining. 
\end{itemize}

\noindent\textbf{Price guide: }\\
\noindent To buy LC 500. Customers must acknowledge that the price includes taxes and other expenses consisting of:

\begin{itemize}
    \item Estimated Price Excl. Govt. Charges (EGC): A\$175,005.00
    \item Luxury Car Tax: A\$29,417.00
    \item Stamp Duty: A\$18,414.00
    \item Transfer Fee: A\$43.70 
\end{itemize}

\noindent\textbf{Second-hand market: }\\
\noindent Life is an unexpected journey. Buyer must check the price of cars in the second-hand market to acknowledge the situation of the price when they sell their car second-hand according to \cite{Ref3.3}.  
\pagebreak
%%%%%%%%%%%%%%%%%%%%%%%%%%%%%%%%%%%%%%%%%%%%%%%%%%%%%%%%%%%%%%%%%%%%%%%%%%%%%%

\setcounter{page}{5}
\section{Question 4}
\subsection{Comparative Report on Amazon and Temu E-commerce Platforms}
\label{sec:Question 4}

\noindent The landscape of e-commerce has dramatically transformed the shopping habits of contemporary consumers. This report presents a comparative analysis between two prominent e-commerce platforms, Amazon and Temu, focusing on their capabilities, variety of electronic payment methods available, security levels, speed, cost, and convenience \citep{Ref4.1}.\\

\noindent\textbf{Website Capabilities and User Experience: }
\begin{itemize}
    \item Amazon, established as the world’s leading e-commerce platform, offers a vast array of products ranging from technology and books to groceries and fashion. It is renowned for its sophisticated search algorithms, personalized recommendations, and Prime membership benefits including fast shipping and exclusive access to entertainment.
    \item Temu, a newer entrant in the market, operates under Pinduoduo Inc., one of China's largest e-commerce firms. Temu has quickly gained popularity for offering a wide range of products at competitive prices, focusing on fashion, home goods, and electronics. It differentiates itself with aggressive pricing strategies and a direct-from-manufacturer model, reducing middleman costs.
\end{itemize}

\noindent\textbf{Electronic Payment Methods:}\\
\noindent Both platforms support a variety of payment methods, enhancing their accessibility and convenience.
\begin{itemize}
    \item Amazon accepts major credit and debit cards, Amazon gift cards, and direct bank payments. In some regions, it also offers Amazon Pay, integrating users' Amazon accounts with external merchant websites.
    \item Temu provides a similar range of payment options, including credit and debit cards and PayPal, which is particularly notable for its buyer protection policies.
\end{itemize}

\noindent\textbf{Security Level:}\\
\noindent Security is paramount in e-commerce to protect user data and financial information.
\begin{itemize}
    \item Amazon employs robust encryption standards and two-factor authentication (2FA) for account security. It adheres to the Payment Card Industry Data Security Standard (PCI DSS) and is known for its strong customer trust and privacy policies \citep{Ref4.2}.
    \item Temu, while newer and less tested in the global market, claims to follow similar encryption practices and security measures. However, as a newer platform, it may not yet have the same level of trust and security reputation as Amazon.
\end{itemize}

\noindent\textbf{Speed, Cost, and Convenience:}
\begin{itemize}
    \item Amazon is famous for its speedy delivery, especially for Prime members, who enjoy same-day or two-day shipping on many items. While its prices are competitive, Amazon’s value is significantly enhanced by its customer service, easy return policies, and the convenience of a one-stop shop for diverse needs.
    \item Temu competes primarily on cost, offering lower prices by connecting consumers directly with manufacturers. Shipping times can vary significantly, often longer than Amazon, due to the international nature of its supply chain. The platform is designed for cost-conscious consumers willing to wait for international shipping to maximize savings.
\end{itemize}

\noindent\textbf{Conclusion: }\\
\noindent In conclusion, Amazon and Temu cater to slightly different consumer needs and preferences. Amazon excels in user experience, security, and speedy delivery but often at a higher cost. Temu, on the other hand, offers competitive pricing and a wide product range, appealing to price-sensitive consumers without immediate delivery needs. Consumers prioritizing convenience and fast shipping might lean towards Amazon, while those seeking the best deals and are comfortable with longer wait times might prefer Temu \citep{Ref4.3}. As the e-commerce landscape continues to evolve, both platforms play significant roles in shaping consumer choices and expectations. 

\pagebreak
%%%%%%%%%%%%%%%%%%%%%%%%%%%%%%%%%%%%%%%%%%%%%%%%%%%%


\setcounter{page}{7}
\section{Question 5}
\subsection{Technological Evolution and Persistent Challenges in E-Commerce}
\label{sec:Question 5}
Technological advancements are steadily overcoming several limitations within the e-commerce sector. Echoing the sentiments of \citep{Ref1.4}, the continuous progression of technology suggests an imminent alleviation of certain technological constraints that e-commerce platforms currently face. The optimism for overcoming these limitations is based on a historical pattern wherein technological innovation and iterative enhancement resolve previous barriers. Challenges such as bandwidth limitations and complexities in systems integration, once considerable obstacles, are being gradually surmounted by improvements in network infrastructure and the development of advanced software interoperability solutions.\\

\noindent Adding to this, compatibility issues with mobile devices are also expected to diminish as a technological limitation. The increasing focus on mobile-first design philosophies and the development of adaptive, responsive web interfaces are aimed at optimizing e-commerce platforms for a variety of screen sizes and operating systems, thus improving the mobile shopping experience.\\

\noindent In contrast, non-technological limitations may not be as easily dispelled. These include the persistent concerns of security and privacy, as well as the desire for a personal touch in shopping experiences—factors that are deeply ingrained in human behavior and social expectations. Despite advances in technology offering more sophisticated security measures, the nature of cyber threats continues to evolve, making it a moving target that requires ongoing attention. Furthermore, cultural and behavioral resistance presents additional challenges. The ingrained habits of consumers and their preference for tactile shopping experiences are cultural elements that influence e-commerce adoption and are not readily adaptable to rapid change. \\
  
\noindent Therefore, while technological innovation is anticipated to reduce the digital barriers in e-commerce, non-technological challenges will likely necessitate more comprehensive and multifaceted strategies to resolve fully. \\

\noindent In conclusion, the future landscape of e-commerce is one where technological enhancements address current operational limitations, making the platforms more robust and user-friendly. However, addressing the non-technological limitations will require an understanding of the nuanced interplay between technology, human behavior, and societal norms. 

\pagebreak
% BIBLIOGRAPHY %%%%%%%%%%%%%%%%%%%%%%%%%%%%%%%%
 

% Use Leeds Harvard referencing template
\bibliographystyle{lsharvard}
% Add here the bib file with your references
\bibliography{references}
	
\def\UrlBreaks{\do\/\do-}

\clearpage
\end{document}
